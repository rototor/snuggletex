\pageId{samples}

\section*{LaTeX Samples}

The pages linked below show some examples of the LaTeX commands that are supported
and the outputs expected. It also demonstrates some of the preset web page outputs that
can be generated.

\emph{Note:} Your browser may not be capable of displaying all of these outputs, though
all browsers should have no problems with the "Legacy" view. See
below for more details.

\begin{tabular}{|c|c|c|c|c|c|}
\hline
Page & Legacy & Mozilla & IE/MathPlayer & USS & Source \\
\hline
Basic Text Mode Commands &
  \href[.html]{text-mode.html} &
  \href[.xhtml]{text-mode.xhtml} &
  \href[.htm]{text-mode.htm} &
  \href[.xml]{text-mode.xml} &
  \href[.tex]{text-mode.tex} \\
\hline
Basic Math Mode Commands &
  \href[.html]{math-mode.html} &
  \href[.xhtml]{math-mode.xhtml} &
  \href[.htm]{math-mode.htm} &
  \href[.xml]{math-mode.xml} &
  \href[.tex]{math-mode.tex} \\
\hline
XML-related Commands &
  \href[.html]{xml-commands.html} &
  \href[.xhtml]{xml-commands.xhtml} &
  \href[.htm]{xml-commands.htm} &
  \href[.xml]{xml-commands.xml} &
  \href[.tex]{xml-commands.tex} \\
\hline
\end{tabular}

\subsection*{About these Outputs}

The outputs above illustrate some of the pre-defined web page output
types described in the \href[Outputs]{outputs.xml} page.

\begin{itemize}
  \item \textbf{Legacy}: Legacy output with XHTML with Maths represented
  by XHTML and/or images. This should display fine on all browsers.

  \item \textbf{Mozilla}: XHTML + MathML for Mozilla/Firefox and friends.
  This will not display at all on IE; other browsers will mess up the MathML.

  \item \textbf{IE/MathPlayer}: HTML designed to work correctly on Internet
  Explorer (provided the MathPlayer plugin is installed). This will not
  display correctly on other browsers.

  \item \textbf{USS}: XHTML designed to work on both Mozilla/Firefox and Internet
  Explorer 6/7 via the ``Universal StyleSheet''. IE users will be prompted to
  download and install MathPlayer if they haven't already done so.

  \item \textbf{Source}: LaTeX source (often somewhat cryptic due to the huge number
  of badly-named commands defined to make the repetitive nature of writing demo pages
  a bit easier!)
\end{itemize}
