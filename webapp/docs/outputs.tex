\pageId{outputs}

\section*{Outputs}

Once you have created a \verb|SnuggleSession| and parsed one or more
\href[\verb|SnuggleInput|]{inputs.html}, you can create one (or more) outputs
using various methods in \verb|SnuggleSession|.

The types of outputs supported by SnuggleTeX fall into 2 basic categories:

\begin{enumerate}
  \item \textbf{DOM Outputs}: These create standalone branches within a DOM.
    Use these if you just want to get an XML ``fragment'' or want to import
    SnuggleTeX output into an existing DOM Docment.

  \item \textbf{Web Page Outputs}: These generate full DOM Documents and/or
    web pages, providing programmaticf control over how the process
    works.
\end{enumerate}

\subsection*{DOM Outputs}

There are two main groups of methods generating this type of output:

\begin{itemize}

  \item The \verb|SnuggleSession.buildXMLString()| methods create an XML fragment (or, more
    formally, a well-formed external general parsed entity), serialized as a
    UTF--8-encoded String.

  \item The \verb|buildDOMSubtree()| methods either append DOM \verb|Node|s representing
    your converted content to an existing \verb|Element| or return a \verb|NodeList|.

\end{itemize}

The results of these methods can be configured using a \verb|DOMOutputOptions| Object,
which is a simple JavaBean that lets you control aspects such as:

\begin{itemize}
  \item Whether to keep LaTeX comments or not;
  \item Prefixing of MathML elements;
  \item Whether to annotate MathML with the LaTeX input;
  \item Whether and how to report LaTeX errors within the XML output;
  \item How to remap any URLs discovered in documents (e.g. with the \verb|\href| command);
  \item Whether to inline CSS using \verb|style| attributes. (This can be useful if your XML
    is going to end up inside some kind of XML application that doesn't support user-specified
    CSS stylesheets.)
  \item (\textbf{Experimental!}) Whether to attempt to down-convert simple MathML expressions
    into plain old XHTML and CSS where deemed possible.
\end{itemize}

Have a look at the API documentation for \verb|DOMOutputOptions| for more information.

\subsection*{Web Page Outputs}

There are two main groups of methods for generating fully-blown web pages:

\begin{itemize}

  \item The \verb|createWebPage()| methods return a DOM \verb|Document|
    representing a web page rendition of your input. You can then serialize this or
    perform further transforms/manipulations as required.

  \item The \verb|writeWebPage()| methods create and write out a web page
    rendition of your input to the given output destination.

\end{itemize}

The results of these methods can be configured by passing an "options" Object, the
final type of which is used to determine the exact process that is followed:

\begin{itemize}
  \item Use a \verb|MathMLWebPageOptions| if you want to generate a web
    page containing XHTML and MathML.

  \item \textbf{Experimental!} Use a \verb|JEuclidWebPageOptions| if you
    want to generate a "legacy" web page consisting of HTML with MathML converted
    to images using the JEuclid tool.
\end{itemize}

\subsubsection*{XHTML + MathML Web Pages}

The \verb|MathMLWebPageOptions| Object extends \verb|DOMOutputOptions|,
giving you additional control over the further aspects of the output:

\begin{itemize}
  \item Which ``page type'' to use for the resulting web page;
  \item Specify client-side XSLT stylesheets to include via \verb|<?xml-stylesheet?>|;
  \item Specify client-side CSS stylesheets to link to via \verb|<link/>|;
  \item Set a language, encoding and title for the resulting page;
  \item Whether to add an automatic title heading element to the web page body;
  \item Whether to indent the output;
  \item Whether to apply your own XSLT Stylesheet (passed as a \verb|Transformer|) to
    the resulting page before it is serialized. (This is useful if you want to add
    in custom headers and footers or otherwise soup up the outputs you get.)
\end{itemize}

Because generating and serving MathML can be difficult and error-prone, SnuggleTeX lets
you choose a \verb|WebPageType| to use as a "template" for your output, which will tweak
the results to suit the particular clients you are targeting. Details are given below;
it is also worth considering the \href[browser requirements]{browser-requirements.html}
page which considers them from a client-specific point of view:

\begin{itemize}
  \item \textbf{MOZILLA}: This generates an XHTML + MathML document suitable for
    Mozilla-based browsers (e.g. Firefox). It is served as \verb|application/xhtml.html|,
    has no XML declaration and no DOCTYPE declaration. Do \emph{not} use this if you
    are targeting Internet Explorer as it will display this as an XML tree.

  \item \textbf{MATHPLAYER\_HTML}: This generates an HTML document that displays well
    on Internet Explorer 6/7 with the MathPlayer plugin installed. It will not display
    MathML correctly on Mozilla-based browsers.

  \item \textbf{CROSS\_BROWSER\_XHTML}: This generates an XHTML + MathML document that
    displays well on both Mozilla-based browsers and Internet Explorer 6/7 (providing that
    the MathPlayer plug-in has already been installed). It is served as \verb|application/xhtml.html|,
    has an XML declaration and a DOCTYPE declaration.
    \begin{itemize}
      \item This will not display correctly on Internet Explorer 6/7 if the
        client has not already installed MathPlayer.
      \item Internet Explorer 6/7 will download the DTD, which will slow rendering down
        somewhat.
    \end{itemize}

  \item \textbf{UNIVERSAL\_STYLESHEET}: This generates an XHTML + MathML document
    that is served as XML and can be served to both Mozilla-based browsers and
    Internet Explorer 6/7. It includes an XML processing instruction that tells
    browsers to apply the Universal MathML StyleSheet to the page before delivering,
    prompting IE users to download and install MathPlayer if they do not already have it.
    \begin{itemize}
      \item IE requires that client-side XSLT stylesheets are loaded from the same server
        that the document came from, so you must copy the USS bundled with SnuggleTeX to
        your own server and tell SnuggleTeX where you put it by calling the
        \verb|setClientSideXSLTStylesheetURLs()| method.
      \item This can be a very good option for decent portability, though it does slow
        rendering down somewhat and may also increase the load on your server as some
        browsers will load static resources both before and after the XSLT is applied.
    \end{itemize}

  \item \textbf{CLIENT\_SIDE\_XSLT\_STYLESHEET}: This generates a XHTML + MathML document,
    served as \verb|application/xhtml.html| with no XML declaration and no DOCTYPE declaration.
    It is intended to be used with a client-side XSLT to invoke something like the
    Universal StyleSheet.

\end{itemize}

\subsubsection*{Legacy (X)HTML + Images Web Pages}

You will need to ensure you have downloaded the full binary distribution of
SnuggleTeX for this, and add all of the JARs under the \verb|lib/jeuclid| folder
to your runtime ClassPath. This will allow you to create PNG images safely; to
create other types of images you will need to add additional JARs as described in
the JEuclid documentation.

The \verb|JEuclidWebPageOptions| allows you to control the output you
get from this. It extends \verb|DOMOutputOptions|, giving you many of the
options that \verb|MathMLWebPageOptions| gives you. Additionally, you will
need to implement and pass a \verb|MathMLImageSavingCallback| to it to control where
the resulting images will be saved to.

This feature is still experimental\ldots though the pages you are now reading
are generated using it!
