\pageId{advanced}
\subsection*{Using a \verb|SnuggleSnapshot|}

It is possible to create a ``snaphsot'' of the state
of a \verb|SnuggleSession|, allowing you to recreate a fresh
session having the same state later on. This is very useful if you want
to reuse ``macro definition'' inputs that define commands and environments. You can
do this with:

\begin{verbatim}SnuggleSnapshot snapshot = session.createSnapshot();

// ... later on ...

SnuggleSession refriedSession = snapshot.createSession();
// refriedSession will have the same parsing state as your
// sesion did when you took the snapshot
\end{verbatim}

\subsection*{Using a \verb|StylesheetCache|}

Some of SnuggleTeX's processes --- in particular the
\href[Semantic Up-Conversion]{semantic-upconversion.html}
extensions --- use XSLT to do some of their work. By default,
SnuggleTeX caches all XSLT stylesheets it creates in a simple
hash-based cache called \verb|SimpleStylesheetCache|. If you don't
like this, or use XSLT in your own application and want to control
caching there, you can implement your own \verb|StylesheetCache|
to pass to SnuggleTeX.
