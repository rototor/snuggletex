\pageId{browserRequirements}

\section*{Browser Requirements}

\subsection*{\href[XHTML + MathML Web Page Outputs]{outputs.html#mathml}}

If you use SnuggleTeX to produce XHTML + MathML web pages, the resulting
pages will only be viewable on browsers capable of displaying MathML.

As you will probably know, most browsers in the Mozilla family (e.g.\
Firefox) support and display MathML natively so are in many ways the best
options. (However, the quality of MathML rendering has regressed to some
degree in the recently-released Firefox 3, which is a little disappointing.)

Internet Explorer 6/7 can be targeted very well using the freely available
MathPlayer plug-in, which does a very nice job of displaying MathML. As of
writing, other browsers do not yet support MathML natively so if you need
to target them, you should consider using the Legacy XHTML + images
\href[output]{outputs.html} option that SnuggleTeX provides as an optional
extension.

As you will see on the \href[output]{outputs.html} page, SnuggleTeX offers
a number of ``templates'' that target specific MathML-capable browsers in
various ways. Each has its own pros and cons and you will need to pick
the one that suits your target audience best.

The following table might help:

\begin{tabular}{|c|l|}
\hline
Target Audience & Suggestions \\
\hline
Firefox only &
Use the \href[\verb|MOZILLA|]{outputs.html#mozilla} page type. %
This is the most efficient way of displaying MathML on these browsers. \\
\hline
IE only with MathPlayer pre-installed &
Use the \href[\verb|MATHPLAYER_HTML|]{outputs.html#mathplayer} page type. %
This is the most efficient option in this case, %
but will \emph{not} work if your clients do not %
already have MathPlayer installed. \\
\hline
IE only, not sure if MathPlayer is installed &
If you know that your clients have sufficient privileges and ability to be able to %
install MathPlayer, then you could use the %
\href[\verb|UNIVERSAL_STYLESHEET|]{outputs.html#uss} option %
as that will prompt for installation of MathPlayer if required. %
Otherwise, it might be safer to serve up legacy content, as described below. \\
\hline
Firefox and IE with MathPlayer pre-installed &
The \href[\verb|CROSS_BROWSER_XHTML|]{outputs.html#crossbrowser} page type %
is your best bet here, though %
you could also use \href[\verb|UNIVERAL_STYLESHEET|]{outputs.html#uss}. \\
\hline
Firefox and IE &
As above, if your clients can install MathPlayer then use the %
\href[\verb|UNIVERSAL_STYLESHEET|]{outputs.html#uss} type. %
Otherwise, it might be safer %
to serve up legacy content, as described below. \\
\hline
All browsers &
Your safest bet in this case is to give up trying MathML and %
serve legacy content, as described below. \\
\hline
\end{tabular}

\subsection*{\href[Legacy XHTML + Images Web Page Outputs]{outputs.html#legacy}}

These outputs are fairly typical ``XHTML'' traditionaly (mis-)served as
\verb|text/html|, meaning that pretty much all (graphical) browsers will
display this perfectly well. Any image elements created to represent
mathematics have an \verb|alt| attribute added containing the SnuggleTeX input
that created the mathematics.
