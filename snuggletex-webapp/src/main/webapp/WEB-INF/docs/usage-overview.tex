\pageId{usageoverview}

\section*{Usage Overview}

\subsection*{The \verb|SnuggleEngine|}

To use SnuggleTeX, you first need to create a \verb|SnuggleEngine|:

\begin{verbatim}SnuggleEngine engine = new SnuggleEngine();\end{verbatim}

An instance of a \verb|SnuggleEngine| is thread-safe and can be used
to perform as many conversions as you like. It houses a number of defaults,
options, command and environment definitions that can be set if you want to
perform a number of very similar conversion processes (which is quite common).
See the \href[API Docs]{maven://apidocs/} for more information.

Despite its heavy-weight name, creating and setting up a \verb|SnuggleEngine|
is not expensive so don't worry about it!

\subsection*{The \verb|SnuggleSession|}

A \verb|SnuggleSession| represents a single SnuggleTeX conversion "job",
which typically consists of the following steps:

\begin{enumerate}
  \item Parse one or more \href[\verb|SnuggleInput|s]{inputs.html}.
  \item Generate one or more \href[Outputs]{outputs.html}.
  \item (Check for any \href[errors]{error-reporting.html} in the LaTeX reported
        by the above proceses\ldots)
\end{enumerate}

To create a new \verb|SnuggleSession|, you use:

\begin{verbatim}SnuggleSession session = engine.createSession();\end{verbatim}

This creates a fresh job using the current default settings in your
\verb|SnuggleEngine|. You can then call methods described in
\href[Inputs]{inputs.html}, \href[Error Reporting]{error-reporting.html}
and \href[Outputs]{outputs.html} to do whatever it is you want to do.

See the \href[First Example]{minimal-example.html} for a simple working
example of how everything comes together.

You can discard a \verb|SnuggleSession| once it has done everything
you need. Note that a \verb|SnuggleSession| is \emph{not} thread safe.

\subsection*{Using \verb|SnugleSnapshot|s}

It is possible to create a ``snaphsot'' of the state
of a \verb|SnuggleSession|, allowing you to recreate a fresh
session having the same state later on. This is very useful if you want
to reuse inputs containing definitions of commands and environments. You can
do this with:

\begin{verbatim}SnuggleSnapshot snapshot = session.createSnapshot();

// ... later on ...

SnuggleSession refriedSession = snapshot.createSession();
// refriedSession will have the same parsing state as your
// sesion did when you took the snapshot
\end{verbatim}
