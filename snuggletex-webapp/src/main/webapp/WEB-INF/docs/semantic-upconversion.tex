\pageId{upconversion}

SnuggleTeX 1.1.0 introduces experimental and limited support for generating
more ``semantic'' outputs than its usual display-oriented Presentation MathML
that some users might find interesting to play around with. This notion is
referred to as ``up-converison'' in SnuggleTeX.

These features were added as part of SnuggleTeX's involvement in the
JISC-funded MathAssess project concerned with enhancing existing computer-based
assessment tools aimed at the educational sector for ``service level''
mathematics teaching up to early undergraduate level in the United Kingdom.

These features make it possible to:

\begin{itemize}

\item Generate semantically richer \href[Enhanced Presentation MathML]{pmathml-enhancement.html} from certain SnuggleTeX inputs

\item \href[Generate Content MathML]{content-mathml.html} from certain SnuggleTeX inputs

\item \href[Generate Maxima input]{maxima-input.html} syntax from certain SnuggleTeX inputs

\item Generate the above forms from certain ASCIIMathML inputs

\end{itemize}

It is important to qualify all of the above with phrases like ``limited'' and ``experimental''
as it is indeed not possible to derive meaning from arbitrary LaTeX math inputs or arbitrary
Presentation MathML expressions. This is not difficult to demonstrate. For
example, written mathematical notations tend to be reused for different
purposes and their correct interpretation relies on an understanding of the
underlying context. For example, the symbol $e$ might refer to the exponential
number in some cases whereas, it could also be the identity in a group or have
a number of other meanings. Similarly, the expression $f(x+1)$ could well
represent the application of the function $f$ to $x+1$, whereas it could also
be the product of $f$ and $x+1$. A related difficulty is that mathmetical notations
and conventions are often localised, with different countries favouring particular
notations and symbolic conventions over others.

\subsection*{Supported Input Forms}

The SnuggleTeX up-conversion processes are aimed at the LaTeX expressions
using:

\begin{itemize}
\item ``Traditional'' UK notation and conventions
\item Numbers and identifiers
\item Symbols for basic arithmetic operators
\item Basic unary trigonometric, hyperbolic, logarithmic and exponential functions
\item Some basic n-ary functions like \verb|min|, \verb|max|, \verb|gcd| etc.
\item Implicit products and function arguments
\item Factorials
\item Basic set theory and logic operators
\item Basic relation operators
\item Greek letters
\item Basic symbols (e.g. infinity)
\end{itemize}

\subsection*{Try For Yourself}

You can play around with these ideas yourself using our
\href[MathML Semantic Up-Conversion Demo]{/UpConversionDemo}.

\subsection*{TODO}

\upConversionExample{\verb|\frac{x}{y}|}

{\bf This document is currently just a draft. It will be finished shortly\ldots}

To write about...

\begin{itemize}
\item Failure handling
\item Conventions followed
\end{itemize}

