\pageId{inputs}

\section*{Inputs - the \verb|SnuggleInput| Object}

A \verb|SnuggleInput| Object acts as a placeholder for a source of LaTeX
input. There are a number of different constructors that let you pull in LaTeX
from other types of I/O Objects, currently \verb|String|, \verb|File|,
\verb|InputStream| and \verb|Reader|.

\textbf{NOTE:} SnuggleTeX behaves like traditional LaTeX in that it expects all input
characters to live in the ASCII subset of Unicode.  Characters outside this
range will be replaced with (useless) alternatives and reported as a
\href[\verb|TTEG02|]{error-codes.html#TTEG02} error.

Given a \verb|SnuggleSession| called \verb|session| and a \verb|SnuggleInput|
called \verb|input|, you can instruct SnuggleTeX to parse it with:
\begin{verbatim}session.parseInput(input);\end{verbatim}
You can parse as many \verb|SnuggleInput|s as you like; they will be treated like one
big continuous input. However, each \verb|SnuggleInput| should be balanced in the sense that any
open environments or braces must be closed within the same input, otherwise errors will
be reported.

The \verb|parseInput()| method returns \verb|false| if parsing was terminated because
the \verb|SnuggleSession| was configured to stop on the first LaTeX error and a LaTeX
error was encountered; otherwise it returns \verb|true|.
See \href[Error Reporting]{error-reporting.html} for more information on
managing errors and configuring how they are reported.
