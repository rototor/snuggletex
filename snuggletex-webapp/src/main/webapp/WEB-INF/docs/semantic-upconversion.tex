\pageId{upconversion}

SnuggleTeX 1.1.0 introduces experimental and limited support for generating
more ``semantically rich'' outputs than its usual display-oriented Presentation
MathML that some users might find interesting to play around with. This notion
is referred to as ``semantic enrichment'' (or ``up-conversion'' in SnuggleTeX).

These features were added as part of SnuggleTeX's involvement in the
JISC-funded \href[MathAssess]{http://mathassess.ecs.soton.ac.uk/}
project which was concerned with enhancing existing computer-based assessment
tools aimed at the educational sector for ``service level'' mathematics
teaching up to early undergraduate level in the United Kingdom.

SnuggleTeX's up-conversion extensions make it possible to:

\begin{itemize}

\item \href[Generate semantically richer Presentation MathML]{docs://pmathmlEnhancement} from certain SnuggleTeX inputs

\item \href[Generate Content MathML]{docs://cmathml} from certain SnuggleTeX inputs

\item \href[Generate Maxima input]{docs://maxima} syntax from certain SnuggleTeX inputs

\item Generate the above forms from certain ASCIIMathML inputs

\end{itemize}

It is important to qualify all of the above with phrases like ``limited'' and
``experimental'' as it is generally not possible to derive meaning from
arbitrary LaTeX math inputs or arbitrary Presentation MathML expressions. This
is not difficult to demonstrate. For example, written mathematical notations
tend to be reused for different purposes and their correct interpretation
relies on an understanding of the underlying context. For example, the symbol
$e$ might refer to the exponential number in some cases whereas, it could also
be the identity in a group or have a number of other meanings. Similarly, the
expression $f(x+1)$ could well represent the application of the function $f$ to
$x+1$, whereas it could also be the product of $f$ and $x+1$. A related
difficulty is that mathematical notations and conventions are often localised,
with different countries favouring particular notations and symbolic
conventions over others.

\subsection*{Supported Input Forms}

The SnuggleTeX up-conversion processes are aimed at the LaTeX expressions
using:

\begin{itemize}
\item ``Traditional'' UK notation and conventions
\item Numbers and identifiers
\item Familiar symbols for basic arithmetic operators
\item Basic unary trigonometric, hyperbolic, logarithmic and exponential functions
\item Some basic n-ary functions like \verb|min|, \verb|max|, \verb|gcd| etc.
\item Implicit products and function arguments
\item Factorials
\item Basic set theory and logic operators
\item Basic relation operators
\item Greek letters
\item Basic symbols (e.g. infinity)
\end{itemize}

\subsection*{Try For Yourself}

You can play around with these ideas yourself using our
\href[MathML Semantic Up-Conversion Demo]{docs://upConversionDemo}.

The pages describing the up-conversion process in more detail also have some
pop-up examples showing the process in action.
