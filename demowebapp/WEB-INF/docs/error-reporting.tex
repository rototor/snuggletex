\pageId{errors}

\section*{Error Reporting}

There are three classes of Errors that can be encountered when using SnuggleTeX:

\begin{itemize}

\item \textbf{Input LaTeX Errors}: Any errors detected in the input LaTeX can
  be detected and recorded in a number of ways, as described below.

\item \textbf{Runtime Errors}: Any problems detected due to a mis-configuration
  or mis-deployment (e.g. if SnuggleTeX can't locate an extension Class or can't
  find one of its XSLT stylesheets or properties files) fall into this category.
  They are raised as (unchecked) \verb|SnuggleRuntimeException|s. Your application
  normally won't want (or indeed be able) to deal with these --- they normally
  indicate you need to fix your ClassPath or something similar.

\item \textbf{Logic Errors}: If SnuggleTeX detects something unexpected in its
  internal logic, then it will throw an (unchecked) \verb|SnuggleLogicException|.
  You should not normally see one of these in the wild --- please file a bug report
  if you do. SnuggleTeX has a test suite that aims to prevent these types of errors
  from happening.
\end{itemize}

\subsection*{Input LaTeX Errors}

By default, a \verb|SnuggleTeXSession| keeps going whenever it detects bad or
invalid LaTeX during parsing.
(Obviously, the final results may or may not be very useful in these cases!)

It is also possible to tell SnuggleTeX to "fail fast" and stop on the first error
encountered with:
\begin{verbatim}session.getConfiguration().setFailingFast(true)\end{verbatim}
In this case, \verb|session.parseInput()| returns false if parsing was
terminated by an error and true if parsing was completed successfully for
a given Input.

You can call:
\begin{verbatim}session.getErrors();\end{verbatim}
at any time to get a \verb|List| of \verb|InputError| Objects representing each
LaTeX error detected by the \verb|SnuggleTeXSession| so far.

\textbf{NOTE:} Further errors can be noted during the SnuggleTeX output processes
so you should check errors \emph{after} generating an output as well!

SnuggleTeX error messages are considerably different from what you might be used to in
LaTeX. They are encapsulated as \verb|InputError| Objects. Each has an error
code; these are all enumerated in the \verb|ErrorCode| enumeration class.
SnuggleTeX is also reasonably good at resolving errors that appear as a result
of user-defined commands and environments being substituted.

The \verb|MessageFormatter| class has some convenience methods for nicely formatting
these error messages in various ways.

It is also possible to "inline" errors into the resulting XML outputs by
calling:
\begin{verbatim}DOMBuilderOptions.setErrorOutputOptions(...)\end{verbatim}
You can choose between not including errors (the default),
SnuggleTeX-namespaced XML error elements or pre-formatted XHTML elements.
