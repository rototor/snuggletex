\pageId{cmathml}

\newcommand{\ue}[1]{\upConversionExample{#1}}

SnuggleTeX can attempt to convert input LaTeX to Content MathML by first
creating \href[Enhanced Presentation MathML]{pmathml-enhancement.html} and
then processing that. In many ways, this part of the process is relatively
simple since most of the semantic structure has already been inferred (though
might not necessarily make any sense).

\subsection*{Configurable Assumptions}

SnuggleTeX currently gives you rather limited control over how certain
constructs should be handled. These options include:

\begin{itemize}
\item whether \verb|e| should be treated as the exponential number or not.
\item whether \verb|i| should be treated as the imaginary number.
\item whether \verb|\pi| should be treated as the number $3.141592\ldots$.
\item whether \verb|( ... )| should be assumed to delimit a vector or not.
\item whether \verb|[ ... ]| should be assumed to delimit a list or not.
\item whether \verb|\{ ... \}| should be assumed to delimit a set or not.
\end{itemize}

We hope to improve these features for SnuggleTeX 1.2.

\subsection*{Supported Operators}

\begin{tabular}{|c|c|c|c|}
\hline
LaTeX operator & Application & Content MathML element & Example \\
\hline
\verb|+| & unary or n-ary & \verb|<plus/>| & \ue{\verb|x+1|} \\
\verb|-| & unary or binary & \verb|<minus/>| & \ue{\verb|-a-b|} \\
Any multiplication & n-ary & \verb|<times/>| & \ue{\verb|A\times 3x|} \\
Any division & binary & \verb|<divide/>| & \ue{\verb|1/2/{3\div 4}|} \\
\verb|\vee| & n-ary & \verb|<or/>| & \ue{\verb|A\vee B|} \\
\verb|\wedge| & n-ary & \verb|<and/>| & \ue{\verb|A\wedge B|} \\
\verb|\cup| & n-ary & \verb|<union/>| & \ue{\verb|A\cup B|} \\
\verb|\cap| & n-ary & \verb|<intersect/>| & \ue{\verb|A\cap B|} \\
\verb|\setminus| & binary & \verb|<setdiff/>| & \ue{\verb|A\setminus B\setminus C|} \\
\verb|\lnot| & unary (prefix) & \verb|<not/>| & \ue{\verb|\lnot \lnot A|} \\
\verb|!| & unary (postfix) & \verb|<factorial/>| & \ue{\verb|x!!|} \\
Any mix of relation operators & binary, applied in adjacent pairs & See below & \ue{\verb|1\leq x < y|} \\
\hline
\end{tabular}

\subsubsection*{Notes}

\begin{itemize}
\item
Operators may be left ``unapplied'', e.g. a raw input of \verb|+|
would result in \verb|<plus/>| with no enclosing
\verb|<apply/>|.

\ue{\verb|+|}

\item
Failures will be registered if an operator is used in an inappropriate
context.
\end{itemize}

\subsection*{Supported Relation Operators}

\begin{tabular}{|c|c|c|}
\hline
LaTeX operator & Content MathML element & Example \\
\hline
\verb|=| & \verb|<eq/>| & \ue{\verb|x=1|} \\
\verb|\not=| & \verb|<neq/>| & \ue{\verb|x\not=a|} \\
\verb|<| & \verb|<lt/>| & \ue{\verb|a<b|} \\
\verb|\not<| & \verb|<not>...<lt/>...</not>| & \ue{\verb|a\not<b|} \\
\verb|>| & \verb|<gt/>| & \ue{\verb|a>b|} \\
\verb|\not>| & \verb|<not>...<gt/>...</not>| & \ue{\verb|a\not>b|} \\
\verb|\leq| & \verb|<leq/>| & \ue{\verb|x\leq 1|} \\
\verb|\not\leq| & \verb|<not>...<leq/>...</not>| & \ue{\verb|x\not\leq 1|} \\
\verb|\geq| & \verb|<geq/>| & \ue{\verb|x\geq 1|} \\
\verb|\not\geq| & \verb|<not>...<geq/>...</not>| & \ue{\verb|x\not\geq 1|} \\
\verb|\equiv| & \verb|<equivalent/>| & \ue{\verb|a\equiv b|} \\
\verb|\not\equiv| & \verb|<not>...<equivalent/>...</not>| & \ue{\verb|a\not\equiv b|} \\
\verb|\approx| & \verb|<approx/>| & \ue{\verb|x\approx 1|} \\
\verb|\not\approx| & \verb|<not>...<approx/>...</not>| & \ue{\verb|x\not\approx 1|} \\
\verb.|. & \verb|<factorof/>| & \ue{\verb.a|b.} \\
\verb.\not|. & \verb|<not>...<factorof/>...</not>| & \ue{\verb.a\not|b.} \\
\verb|\in| & \verb|<in/>| & \ue{\verb|a\in A|} \\
\verb|\not\in| & \verb|<notin/>| & \ue{\verb|a\not\in A|} \\
\hline
\end{tabular}

\subsubsection*{Notes}

\begin{itemize}
\item
  Relation operators will be paired up earlier in the up-conversion
  process. Where there are two or more relations together, each pairing
  becomes an operand of an enclosing logical \verb|and| when
  converting to Content MathML. So a LaTeX input like \verb|1<x \leq 2| will
  result in the same output as \verb|(1<x) \land (x \leq 2)|.

\item
  Note that this pairing is still done for inputs like \verb|a=b=c|,
  even though they could have been converted to an $n$-ary application of
  the \verb|<eq/>| operator.
\end{itemize}

\subsection*{Supported Functions}

\begin{tabular}{|c|c|c|c|c|}
\hline
LaTeX function & Arity & Invertible & Content MathML element & Example \\
\hline
\verb|\sin| & unary & Yes & \verb|<sin/>| or \verb|<arcsin/>| & \ue{\verb|\sin x|} \\
\verb|\cos| & unary & Yes & \verb|<cos/>| or \verb|<arccos/>| & \ue{\verb|\cos^{-1} 0|} \\
\verb|\tan| & unary & Yes & \verb|<tan/>| or \verb|<arctan/>| & \ue{\verb|\tan\tan^{-1}x|} \\
\verb|\sec| & unary & Yes & \verb|<csc/>| or \verb|<arcsec/>| & \ue{\verb|\sec 0|} \\
\verb|\cot| & unary & Yes & \verb|<cot/>| or \verb|<arccot/>| & \ue{\verb|\cot x|} \\
\verb|\sinh| & unary & Yes & \verb|<sinh/>| or \verb|<arcsinh/>| & \ue{\verb|\sinh x|} \\
\verb|\cosh| & unary & Yes & \verb|<cosh/>| or \verb|<arccosh/>| & \ue{\verb|\cosh x|} \\
\verb|\tanh| & unary & Yes & \verb|<tanh/>| or \verb|<arctanh/>| & \ue{\verb|\tanh x|} \\
\verb|\sech| & unary & Yes & \verb|<sech/>| or \verb|<arcsech/>| & \ue{\verb|\sech^{-1}x|} \\
\verb|\csch| & unary & Yes & \verb|<csch/>| or \verb|<arccsch/>| & \ue{\verb|\csch x|} \\
\verb|\coth| & unary & Yes & \verb|<coth/>| or \verb|<arccoth/>| & \ue{\verb|\coth x|}  \\
\verb|\arcsin| & unary & No & \verb|<arcsin/>| & \ue{\verb|\arcsin x|} \\
\verb|\arccos| & unary & No & \verb|<arccos/>| & \ue{\verb|\arccos x|} \\
\verb|\arctan| & unary & No & \verb|<arctan/>| & \ue{\verb|\arctan x|} \\
\verb|\arcsec| & unary & No & \verb|<arcsec/>| & \ue{\verb|\arcsec x|} \\
\verb|\arccsc| & unary & No & \verb|<arccsc/>| & \ue{\verb|\arccsc x|} \\
\verb|\arccot| & unary & No & \verb|<arccot/>| & \ue{\verb|\arccot x|} \\
\verb|\arcsinh| & unary & No & \verb|<arcsinh/>| & \ue{\verb|\arcsinh x|} \\
\verb|\arccosh| & unary & No & \verb|<arccosh/>| & \ue{\verb|\arccosh x|} \\
\verb|\arctanh| & unary & No & \verb|<arctanh/>| & \ue{\verb|\arctanh x|} \\
\verb|\arcsech| & unary & No & \verb|<arcsech/>| & \ue{\verb|\arcsech x|} \\
\verb|\arccsch| & unary & No & \verb|<arccsch/>| & \ue{\verb|\arccsch x|} \\
\verb|\arccoth| & unary & No & \verb|<arccoth/>| & \ue{\verb|\arccoth x|} \\
\verb|\ln| & unary & No & \verb|<ln/>| & \ue{\verb|\ln x|} \\
\verb|\log| & unary & No & \verb|<log/>| & \ue{\verb|\log x|} \\
\verb|\exp| & unary & No & \verb|<exp/>| & \ue{\verb|\exp x|} \\
\verb|\det| & unary & No & \verb|<determinant/>| & \ue{\verb|\det A|} \\
\verb|\gcd| & n-ary & No & \verb|<gcd/>| & \ue{\verb|\gcd(x,y)|} \\
\verb|\lcm| & n-ary & No & \verb|<lcm/>| & \ue{\verb|\lcm(x,y)|} \\
\verb|\max| & n-ary & No & \verb|<max/>| & \ue{\verb|\max(1,2,3)|} \\
\verb|\min| & n-ary & No & \verb|<min/>| & \ue{\verb|\min A|} \\
\verb|\Re| & n-ary & No & \verb|<real/>| & \ue{\verb|\Re z|} \\
\verb|\Im| & n-ary & No & \verb|<imaginary/>| & \ue{\verb|\Im(1+3i)|} \\
$\ldots$ & & & \\
\hline
\end{tabular}

\subsubsection*{Notes}

\begin{itemize}
\item
  For all functions, constructs like \verb|\cos^n x| is
  interpreted as ``cosine $x$ raised to the power of $n$''
  where $n$ is either an identifier or a number greater than
  or equal to $1$.

  \ue{\verb|\cos^2x+\sin^2x = 1|}

\item
  For functions listed as Invertible in the table above,
  the up-conversion process interprets constructs like
  \verb|\sin^{-1} x| as the ``inverse sin of $x$''
  and would result in \verb|<apply><arcsin/><ci>x</ci></apply>|.
  A failure will be noted if constructs like these are
  used on functions which do not support this.

  \ue{\verb|\sin^{-1}0 = 0|}

\item
  For the $\log$ function, an input like \verb|\log_a x|
  is interpreted as ``logarithm to base $a$ of $x$''.

  \ue{\verb|\log_{10}100 = 2|}
\end{itemize}

\subsection*{Supported Symbols}

\begin{tabular}{|c|c|c|}
\hline
LaTeX symbol & Content MathML interpretation & Example \\
\hline
\verb|\emptyset| & \verb|<emptyset/>| & \ue{\verb|A=\emptyset|} \\
\verb|\infty| & \verb|<infinity/>| & \ue{\verb|x<\infty|} \\
\hline
\end{tabular}

\subsection*{Other Supported Constructs}

\begin{itemize}

\item
  Powers are supported in the expected way. The special case of \verb|e^x|
  is normally treated as \verb|\exp x|. This can be disabled if desired.

  \ue{\verb|e^{ab} = e^ae^b|}

\item
  Square and $n$th roots are supported using the familiar LaTeX
  constructs \verb|\sqrt{x}| and \verb|\sqrt[n]{x}|.

  \ue{\verb|\sqrt\sqrt{x} = \sqrt[4]{x}|}

\item
  Subscripted identifiers like \verb|x_1|, \verb|x_{1,2}|
  and \verb|a_{x_y}| are supported and are kept as presentation MathML wrapped
  inside a \verb|<ci/>| container element.

  \ue{\verb|A_{n,i}x_i|}

\item
  SnuggleTeX includes a custom macro called \verb|\units| that can be used
  to denote that its argument represent units. The result of up-converting
  this is a \verb|<csymbol/>| element.

  \ue{\verb|5\units{kg}|}

\end{itemize}


