\pageId{advanced}
\subsection*{Using a \verb|SnuggleSnapshot|}

It is possible to create a ``snaphsot'' of the state
of a \verb|SnuggleSession|, allowing you to recreate a fresh
session having the same state later on. This is very useful if you want
to reuse ``macro definition'' inputs that define commands and environments. You can
do this with:

\begin{verbatim}SnuggleSnapshot snapshot = session.createSnapshot();

// ... later on ...

SnuggleSession refriedSession = snapshot.createSession();
// refriedSession will have the same parsing state as your
// sesion did when you took the snapshot
\end{verbatim}

\subsection*{Using a \verb|StylesheetCache|}

Some of SnuggleTeX's processing --- in particular the
\href[Semantic Up-Conversion]{semantic-upconversion.html}
processes --- use XSLT to do the work. In order to make this
as performant as possible, SnuggleTeX includes a simple
interface called \verb|StylesheetCache| supporting basic
caching of stylesheets, as well as a trivial hash-based
implementation called \verb|SimpleStylesheetCache|. If your
own application uses XSLT and does its own caching, you should
be able to create your own implementation of \verb|StylesheetCache|
that uses your own caching mechanism. In those cases, use the
constructors for certain SnuggleTeX objects that take an implementation
of \verb|StylesheetCache|.
