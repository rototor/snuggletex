\pageId{firstExample}

Look at the \href[\verb|MinimalExample.java|]{maven://xref/uk/ac/ed/ph/snuggletex/samples/MinimalExample.html}.
(The source for this is available in the ``full'' binary distribution if you want to
compile this yourself.)
file in the full binary distribution.
This demonstrates a very simple example of calling up SnuggleTeX:

\begin{verbatim}
/* Create vanilla SnuggleEngine and new SnuggleSession */
SnuggleEngine engine = new SnuggleEngine();
SnuggleSession session = engine.createSession();

/* Parse some very basic Math Mode input */
SnuggleInput input = new SnuggleInput("$$1+2=3$$");
session.parseInput(input);

/* Convert the results to an XML String, which in this case will
 * be a single MathML <math>...</math> element. */
String xmlString = session.buildXMLString();
System.out.println("Input " + input.getString()
        + " was converted to:\n" + xmlString);
\end{verbatim}

This block of code converts the LaTeX String \verb|$$1+2=3$$| to a simple XML
String fragment, which is printed to the standard output.
For this particular input, the resulting XML fragment consists of a single MathML
\verb|math| element.

You can run this on the command line with:

\begin{verbatim}java -classpath snuggletex-core-n.n.n.jar \
uk.ac.ed.ph.snuggletex.extensions.samples.MinimalExample\end{verbatim}

(This assumes that \verb|snuggletex-core-n.n.n.jar| is in the current directory; you will
of course need to provide a path if it is not. You should also substitute
\verb|n.n.n| for the actual version number attached to your SnuggleTeX JAR.)

Once run, you should get an output like:

\begin{verbatim}
Input $$1+2=3$$ was converted to:
<math xmlns="http://www.w3.org/1998/Math/MathML" display="block">
  <mn>1</mn><mo>+</mo><mn>2</mn><mo>=</mo><mn>3</mn></math>
\end{verbatim}

The XML fragment output in the above example is great for showing off the outputs
from SnuggleTeX but is not usually much use if you want to do anything else with it.
SnuggleTeX supports a number of alternative \href[outputs]{outputs.html} that can be
more useful to programmers.
