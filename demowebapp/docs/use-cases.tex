\pageId{usecases}

\section*{Use Cases --- Why Use SnuggleTeX?}

There are a healthy number of tools which have functionality related to SnuggleTeX,
which can make it hard to decide which tool bits fits your given requirements.
Like all tools, SnuggleTeX has its relative strengths and weaknesses that you
should consider:

\begin{itemize}

  \item SnuggleTeX is 100\% Java with minimal (usually no) dependencies on
        other libraries so can be easily integrated into a Java software
        development project as a library for converting LaTeX to XML.

  \item SnuggleTeX was originally developed to support
        \href[Aardvark]{http://www.ph.ed.ac.uk/elearning/projects/aardvark/}
        in order to facilitate the conversion of fragments of LaTeX written by
        academics into XML tree branches. Another example of this type of use
        would be a kind of LaTeX-based Wiki, where SnuggleTeX is used to
        convert the LaTeX and graft it into a final web page.

  \item SnuggleTeX can also be used to generate "legacy" web pages where
        mathematical forumalae are represented by HTML + CSS (if suitably
        simple) and/or images. This uses the open-source JEuclid library.

  \item SnuggleTeX was \emph{not} intented to be a standalone tool that you
        could throw complete LaTeX documents at and have them converted to web
        pages. Other tools do this type of thing very well.

  \item SnuggleTeX supports a usable subset of LaTeX but does not include anything
        which is particularly paper-- or page--specific. It also currently doesn't
        do cross-referencing or numbering.

  \item SnuggleTeX's parser pretends that TeX never happened and may behave slightly
        differently to what experienced LaTeX users might expect. Novice LaTeX users
        will not notice any difference and might find the error messages provided
        more useful.

\end{itemize}
