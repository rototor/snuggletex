\pageId{inputs}

\section*{Inputs - the \verb|SnuggleInput| Object}

A \verb|SnuggleInput| Object acts as a place-holder for a source of LaTeX
input. There are a number of different constructors that let you pull in LaTeX
from other types of I/O Objects (currently \verb|String|, \verb|File|,
\verb|InputStream| and \verb|Reader|).

\textbf{NOTE:} SnuggleTeX behaves like traditional LaTeX in that it expects all input
characters to live in the ASCII subset of Unicode.  Characters outside this
range will be replaced with (useless) alternatives and reported as an error.

Given a \verb|SnuggleTeXSession| called \verb|session| and a \verb|SnuggleInput|
called \verb|input|, you can parse it with:
\begin{verbatim}session.parseInput(input);\end{verbatim}
You can parse as many \verb|SnuggleInput|s as you like; they will be treated like one
big input. (However, each \verb|SnuggleInput| should be balanced in the sense that any
open environments or braces must be closed within the same input, otherwise errors will
be reported.)

The \verb|parseInput()| method returns \verb|true| if it succeeded without finding
any errors in the LaTeX input, otherwise it returns \verb|false|. See
\href[Error Reporting]{error-reporting.html} for more information on managing errors
and configuring how they are reported.
