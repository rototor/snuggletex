\pageId{webPageTypes}

SnuggleTeX supports a number of pre-defined ``Web Page Types'' that you can select
in order to generate MathML-based web pages that work across a number of different
browsers and platforms.

These types are defined in the \verb|WebPageOutputOptions.WebPageType|
enumeration and you simply pass one of these enumerated values to the
\verb|WebPageOutputOptionsTemplates.createWebPageOptions()| utility method to
create a working \verb|WebPageOutputOptions| Object that you can tweak or use
immediately.

(You might also want to look at the \href[MathML Browser Requirements]{browser-requirements.html}
page, which looks at these ideas from a slightly different perspective.)

\subsection*{Pre-defined \verb|WebPageType|s}

\subsubsection*{\anchor{MOZILLA}\verb|MOZILLA|}
This generates an XHTML + MathML document suitable for
Mozilla-based browsers (e.g. Firefox). It is served as \verb|application/xhtml.html|,
with no XML declaration and no DOCTYPE declaration. Do \emph{not} use this if you
are targeting Internet Explorer as it will display your page as an XML tree.

\href[Example]{math-mode.xhtml}

\subsubsection*{\anchor{crossbrowser}\verb|CROSS_BROWSER_XHTML|}

This generates an XHTML + MathML document that displays well on both
Mozilla-based browsers and Internet Explorer 6 and above (providing that the
MathPlayer plug-in has already been installed).  It is served as
\verb|application/xhtml.html|, has an XML declaration and a DOCTYPE
declaration.
\begin{itemize}
  \item This will not display correctly on Internet Explorer if MathPlayer has not
    already been installed on it by your target user.
  \item Internet Explorer will download the DTD, which will slow rendering down
    somewhat.
\end{itemize}

\href[Example]{math-mode.cxml}

\subsubsection*{\anchor{mathplayer}\verb|MATHPLAYER_HTML|}

This generates an HTML document that displays well on Internet Explorer 6 and
above with the MathPlayer plugin already installed. It will not display MathML
correctly on Mozilla-based browsers.

\href[Example]{math-mode.html}

\subsubsection*{\anchor{uss}\verb|UNIVERSAL_STYLESHEET|}

This generates an XHTML + MathML document that is served as XML and can be
served to both Mozilla-based browsers and Internet Explorer 6+. It includes an
XML processing instruction that tells browsers to apply the Universal MathML
StyleSheet to the page before delivering, prompting IE users to download and
install MathPlayer if they do not already have it.
\begin{itemize}
  \item IE requires that client-side XSLT stylesheets are loaded from the same server
    that the document came from, so you must copy the USS bundled with SnuggleTeX to
    your own server and tell SnuggleTeX where you put it by calling the
    \verb|setClientSideXSLTStylesheetURLs()| method.
  \item This can be a very good option for decent portability, though it does slow
    rendering down somewhat and may also increase the load on your server as some
    browsers will load static resources both before and after the XSLT is applied.
\end{itemize}

\href[Example]{math-mode.xml}

\subsubsection*{\anchor{xsl}\verb|CLIENT_SIDE_XSLT_STYLESHEET|}

This generates a XHTML + MathML document,
served as \verb|application/xhtml.html| with no XML declaration and no DOCTYPE declaration.
It is intended to be used with a client-side XSLT of your choice.
(This is a more advanced option!)

\subsubsection*{\anchor{processed}\verb|PROCESSED_HTML|}

This is another advanced option that assumes it will generate a plain old
HTML document, served as \verb|text/html|.
You will have to post-process the resulting DOM to produce something sensible here
by registering a \verb|DOMPostProcessor| with your \verb|WebPageOutputOptions|.
This feels like a rather esoteric option, but is actually where the SnuggleTeX JEuclid
module hooks in when converting the results to legacy HTML + Images.
