\pageId{upconversionFailures}

\newcommand{\ue}[1]{\upConversionExample{#1}}

As we have mentioned, the up-conversion process only works with certain LaTeX
inputs. If the process cannot convert a certain input to Content MathML or Maxima,
then SnuggleTeX continues as normal, recording information about why it cannot perform
the up-conversion as an annotation in the resulting MathML.

\subsection*{Failure to generate Content MathML}

Consider the following example:

\ue{\verb|a==2|}

While this might make sense in certain programming languages, the up-conversion
process has not been taught to handle two adjacent relation operators so fails
to convert this input to Content MathML.

The result will be an annotated MathML element containing the (enhanced)
Presentation MathML and a special \verb|<annotation-xml/>| element hopefully
revealing where the conversion to Content MathML failed:

\begin{verbatim}
<math xmlns="http://www.w3.org/1998/Math/MathML" display="block">
   <semantics>
      <mrow>
         <mn>1</mn>
         <mo>=</mo>
         <mrow>
            <mo>=</mo>
            <mn>2</mn>
         </mrow>
      </mrow>
      <annotation-xml encoding="MathML-Content-upconversion-failures">
         <s:fail xmlns:s="http://www.ph.ed.ac.uk/snuggletex" code="UCFOP4"
                 message="Unsupported grouping of relation operators">
            <s:context>
               <mo>=</mo>
               <mn>2</mn>
            </s:context>
         </s:fail>
      </annotation-xml>
   </semantics>
</math>
\end{verbatim}

The failure code(s) will all have the form \verb|UCFxxx| when up-conversion
to Content MathML fails. These are listed in the
\href[SnuggleTeX Error Codes]{error-codes.html} page. (Do however note that
these are not reported the same way as parsing or DOM-building errors.)

\subsection*{Failure to generate Maxima Input}

Generating Maxima Input syntax requires the conversion to Content MathML to
succeed, so if that part of the process fails then so will the conversion to
Maxima Input syntax. On the other hand, many inputs will generate Content MathML
which cannot then be converted into a sensible Maxima form.

Consider the following example:

\ue{\verb|1_a|}

This generates the following MathML:

\begin{verbatim}
<math xmlns="http://www.w3.org/1998/Math/MathML" display="block">
   <semantics>
      <msub>
         <mn>1</mn>
         <mi>a</mi>
      </msub>
      <annotation-xml encoding="MathML-Content">
         <ci>
            <msub>
               <mn>1</mn>
               <mi>a</mi>
            </msub>
         </ci>
      </annotation-xml>
      <annotation-xml encoding="Maxima-upconversion-failures">
         <s:fail xmlns:s="http://www.ph.ed.ac.uk/snuggletex" code="UMFG04"
                 message="Cannot create subscripted variable">
            <s:context>
               <ci>
                  <msub>
                     <mn>1</mn>
                     <mi>a</mi>
                  </msub>
               </ci>
            </s:context>
         </s:fail>
      </annotation-xml>
   </semantics>
</math>
\end{verbatim}

Here, the Maxima-related failures are contained within an
\verb|<annotation-xml/>| element with encoding \verb|Maxima-upconversion-failures|
and all have codes of the form \verb|UMFxxx|.

\subsection*{Extracting Up-Conversion Failure Information}

You can extract details of any up-conversion failures from an annotated MathML
\verb|Element| or \verb|Document| using the utility method
\verb|UpConversionUtilities.extractUpConversionFailures()|. This returns
a \verb|List| or \verb|UpConversionFailure| Objects than can be interrogated or
converted to error messages.
