\pageId{webOutput}

SnuggleTeX tries to make it easy to create web pages suitable for use
on various browser platforms, creating XHTML+MathML web pages by default.
It can also generate \href[Legacy Web Pages]{docs://legacyWebPages} via
the optional JEuclid extension.

\subsection*{Usage Recipe}

\begin{enumerate}

\item
  You must first decide what kind of web page you would like to generate.
  These are listed in the \href[XHTML+MathML Web Page Types]{docs://webPageTypes}
  page. (You might prefer to create a \href[Legacy Web Page]{docs://legacyWebPages}.)

\item
  Pass your selected \verb|WebPageType| to
  \verb|WebPageOutputOptionsTemplate.createWebPageOptions()|. This will
  create a \verb|WebPageOutputOptions| Object configured with suitable defaults
  for the type of web page you want to create.

\item
  You can now tweak your newly-created \verb|WebPageOutputOptions| Object to control
  the following aspects of the output:
  \begin{itemize}
    \item Set a language, encoding and title for the resulting page;
    \item Specify client-side XSLT stylesheets to include via \verb|<?xml-stylesheet?>|;
    \item Specify client-side CSS stylesheets to attach via \verb|<link/>|;
    \item Whether to inline the SnuggleTeX's CSS styles as a \verb|<style/>| section
      or leave you to link to it yourself.
    \item Whether to add an automatic title heading element to the web page body;
    \item Whether to indent the output;
    \item Whether to apply your own XSLT Stylesheets (passed as JAXP \verb|Transformer| Objects)
      to the resulting page before it is serialized. (This is useful if you want to add
      in custom headers and footers or otherwise soup up the outputs you get.)
    \item \ldots plus the same options as for generating \href[XML/DOM Outputs]{docs://domOutput}.
  \end{itemize}

\item
  Finally, you pass your \verb|WebPageOutputOptions| Object to either
  \verb|snuggleSession.createWebPage()| or \verb|snuggleSession.writePageWeb()|
  methods to generate one of the following outputs:
  \begin{itemize}
    \item The \verb|snuggleSession.createWebPage()| methods return a DOM \verb|Document|
      representing a web page rendition of your input. You can then serialize this or
      perform further transforms/manipulations as required.

    \item The \verb|snuggleSession.writeWebPage()| methods create and write out a web page
      rendition of your input to the given \verb|OutputStream|.
  \end{itemize}
\end{enumerate}

\subsection*{Example Code}

Have a look at \href[\verb|WebPageExample.java|]{maven://xref/uk/ac/ed/ph/snuggletex/samples/WebPageExample.html}
for a simple self-contained example.
