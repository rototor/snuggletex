\pageId{xmlCommands}

\section*{XML-related Commands}

SnuggleTeX provides a few custom commands to make it
easy to do certain XML-related tasks, which can be quite
useful if you are generating elements outwith the usual
spectrum of XHTML and/or MathML.

\subsection*{Creating Custom XML Elements}

You can ask SnuggleTeX to create custom XML using the
\verb|\xmlBlockElement| and \verb|\xmlInlineElement|
commands. Both output exactly the same XML elements, however the former will
output the XML as a new ``block'' whereas the latter will output the XML
``inline'' with whatever surrounds it.

Both commands take 3 required arguments, plus an
optional one to allow you to set XML attributes
(see below). The required arguments are as follows:

\begin{enumerate}
  \item Namespace URI. This must be a valid URI, otherwise
    error \href[\verb|TDEX04|]{error-codes.html#TDEX04} will be recorded.
    (Note that this argument is parsed in LR mode
    rather than \verb|VERBATIM| mode so as to allow you to
    parametrise it, if required.)
  \item Element qName (qualified name), which will be of
    the form \verb|prefix:localName| or just
    \verb|localName|. An invalid qName will result in
    error \href[\verb|TDEX01|]{error-codes.html#TDEX01}. (This argument is also parsed
    in LR mode.)
  \item Element content. (LR mode)
\end{enumerate}

If you are generating a lot of XML elements belonging
to a particular namespace, you will probably want to
define macros (either via \verb|\newcommand| or
through the Java API) that fill in more of the details.

\subsubsection*{Example}

\begin{verbatim}
\newcommand{\xhtmlBlockElement}[2][]
  {\xmlBlockElement[#1]{http://www.w3.org/1999/xhtml}{#2}{#3}}
\xhtmlBlockElement{hr}{}
\end{verbatim}

This generates an XHTML \verb|<hr/>| element:

\newcommand{\xhtmlBlockElement}[3][]
  {\xmlBlockElement[#1]{http://www.w3.org/1999/xhtml}{#2}{#3}}
\xhtmlBlockElement{hr}{}

\subsection*{XML Attributes}

The \verb|\xmlBlockElement| and \verb|\xmlInlineElement| commands both take
optional arguments that allow XML attributes to be specified. This is
done using a number of \verb|\xmlAttr| commands. These take 3 arguments
--- namespace URI, qName and value --- in the same way as above. Using
an \verb|\xmlAttr| in any other place in a document will record a
\href[\verb|TDEX02|]{error-codes.html#TDEX02} error.

\subsection*{XML Names and IDs}

It is sometimes useful to be able to enforce that a particular input
is a valid XML name. This can be done with the \verb|\xmlName|
and \verb|\xmlName*| commands. These both take a single argument and returns it
unchanged, recording a \href[\verb|TDEX03|]{error-codes.html#TDEX03} error if the name is invalid.
The \verb|\xmlName| parses its argument in VERBATIM Mode, saving users from
having to escape the underscore character. However, this also makes it
impossible to dynamically evaluate the names. For this, you can use
\verb|\xmlName*|, which parses its argument in LR Mode.

The \verb|\xmlId| and \verb|\xmlId*| commands behave similarly, performing
an additional check that the name is an ID which is not already in use in the
resulting DOM Document, recording a \href[\verb|TDEX05|]{error-codes.html#TDEX05} error if this is not the
case.
