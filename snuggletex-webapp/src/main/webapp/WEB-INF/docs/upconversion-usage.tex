\pageId{upconversionUsage}

\subsection*{Getting Started}

\begin{itemize}

\item
  In order to use the up-conversion features, you will have to download the
  full distribution of SnuggleTeX.

\item
  You need to have \verb|snuggletex-core.jar| and
  \verb|snuggletex-upconversion.jar| in your ClassPath, as well as
  \verb|saxon9.jar| and \verb|saxon9-dom.jar|. (The JARs in the distributon
  ZIP file contain version numbers in their names; it should be obvious what
  we mean here.)

\end{itemize}

\subsection*{Example}

Have a look at the source code for
\href[\verb|BasicUpConversionExample.java|]{maven://xref/uk/ac/ed/ph/snuggletex/upconversion/samples/BasicUpConversionExample.html}
to see how to invoke the up-conversion process.

The key here is to attach an \verb|UpConvertingPostProcessor| into the
SnuggleTeX DOM-building process using the \verb|addDOMPostProcessor()| method
of your \verb|DOMOutputOptions| or \verb|WebOutputOptions|
Object.

You can configure how the up-conversion process works by passing some
\verb|UpConversionParameters| to your \verb|UpConvertingPostProcesor|.

\subsection*{Extracting Results}

The up-conversion process {\it replaces} each raw MathML \verb|<math/>| element
with an annotated element housing all of the resulting output forms together. In
the case of the above example, the resulting MathML is:

\begin{verbatim}
<math xmlns="http://www.w3.org/1998/Math/MathML" display="block">
   <semantics>
      <mfrac>
         ...
      </mfrac>
      <annotation-xml encoding="MathML-Content">
         <apply>
            <divide/>
            ...
         </apply>
      </annotation-xml>
      <annotation encoding="Maxima">((2 * x) - (y^2)) / sin((x * y * (x - 2)))</annotation>
   </semantics>
</math>
\end{verbatim}

The SnuggleTeX utility class
\href[\verb|MathMLUtilities|]{maven://xref/uk/ac/ed/ph/snuggletex/utilities/MathMLUtilities.html}
includes some useful helper methods for extracting individual parts of the
MathML. Here are some examples:

\begin{itemize}

\item \verb|extractAnnotationXML(element, "MathML-Content")|
will extract the content of the named \verb|<annotation-xml/>| element,
in this case giving you an \verb|Element| of the form:
\begin{verbatim}
<apply>
   <divide/>
   ...
</apply>
\end{verbatim}

\item You can use \verb|isolateAnnotationXML(element, "MathML-Content")|
to do the same thing, but wrapping the result in a \verb|<math/>| having
the same attributes as the original one, giving:
\begin{verbatim}
<math xmlns="http://www.w3.org/1998/Math/MathML" display="block">
  <apply>
     <divide/>
     ...
  </apply>
</math>
\end{verbatim}

\item Use \verb|extractAnnotationString(element, "Maxima")|
to get the Maxima input annotation. This would give you:
\begin{verbatim}
((2 * x) - (y^2)) / sin((x * y * (x - 2)))
\end{verbatim}

\item The \verb|unwrapParallelMathMLDOM()| method returns a simple
Object with all of the annotations extracted that can be easily
interrogated.

\item
\href[\verb|MathMLUtilities|]{maven://xref/uk/ac/ed/ph/snuggletex/utilities/MathMLUtilities.html}
has some convenience methods for serialising MathML \verb|Element|s and
\verb|Document|s as XML Strings, that can be useful for debugging or
documentation purposes.

\end{itemize}
