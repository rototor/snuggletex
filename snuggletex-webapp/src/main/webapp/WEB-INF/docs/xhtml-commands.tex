\pageId{xhtmlCommands}

SnuggleTeX provides a small number of custom commands to do
certain web-related tasks. (There are also a number of related commands for
doing more general \href[XML-related]{xml-commands.html} things.)

\subsection*{Hyperlinks}

Use the \verb|\href| command to generate a simple web hyperlink. It
takes an optional argument and one required argument as follows:

\begin{itemize}
  \item \textbf{Link Text} (optional): This is the text that users
    will click on to follow the link. If omitted, the supplied URL
    will be displayed instead. This is parsed in LR mode.

  \item \textbf{URL} (required): This is the URL (i.e.\ web address)
    that the hyperlink should go to. This is parsed in verbatim mode.

\end{itemize}

The URL must follow the correct syntax for a URI, otherwise a
\href[\verb|TDEX04|]{error-codes.html#TDEX04} error is recorded.

SnuggleTeX provides a programmatic mechanism for messing around with
URLs before they are output, which can sometimes be useful. Have a look
at the \verb|LinkResolver| interface, which you implement and pass as
part of your \verb|DOMBuilderOptions|.

\subsubsection*{Example}

\begin{verbatim}
\href[SnuggleTeX Home Page]{http://www.ph.ed.ac.uk/snuggletex/}
\end{verbatim}

produces a link to the SnuggleTeX home page.

\subsection*{Anchors}

You can use the \verb|\anchor| command to create a hypertext anchor
within a page. This anchor can be referred to using \verb|\href| with a URL of
the form \verb|...#anchorId|.

\subsubsection*{Example}

\begin{verbatim}
\subsection*{\anchor{boring}A Long And Boring Section}

...

\subsection*{Later On}

Back in the \href[long and boring section]{#boring} we saw...
\end{verbatim}

The \verb|\anchor| command parses its argument in verbatim mode; a variant
command \verb|\anchor*| also exists which parses its argument in LR mode,
which can sometimes be useful.
